\documentclass[%singlesided,
               doublesided,
               paper=a4,
               fontsize=10pt
              ]{my-resume}


%%%%%%%%%%%%%%%%%%%%%%%%%%%%%%%%%%%%%%%%%%%%%%%%%%%%%%%%%%%%%%%%%%%%%%%%%%%%%%%%
% set geometry
%%%%%%%%%%%%%%%%%%%%%%%%%%%%%%%%%%%%%%%%%%%%%%%%%%%%%%%%%%%%%%%%%%%%%%%%%%%%%%%%

\setlength\highlightwidth{8cm}
\setlength\headerheight{4cm}            % note that margintop gets added to this value, i.e. the header bar is 5cm
\setlength\marginleft{1cm}
\setlength\marginright{\marginleft}      % needs to be 1.5 times to be actually equal. why?
\setlength\margintop{1cm}
\setlength\marginbottom{1cm}


%%%%%%%%%%%%%%%%%%%%%%%%%%%%%%%%%%%%%%%%%%%%%%%%%%%%%%%%%%%%%%%%%%%%%%%%%%%%%%%%
% FONTS
%%%%%%%%%%%%%%%%%%%%%%%%%%%%%%%%%%%%%%%%%%%%%%%%%%%%%%%%%%%%%%%%%%%%%%%%%%%%%%%%

\RequirePackage{fontspec}
\setmainfont{Carlito}


%%%%%%%%%%%%%%%%%%%%%%%%%%%%%%%%%%%%%%%%%%%%%%%%%%%%%%%%%%%%%%%%%%%%%%%%%%%%%%%%
% COLORS
%%%%%%%%%%%%%%%%%%%%%%%%%%%%%%%%%%%%%%%%%%%%%%%%%%%%%%%%%%%%%%%%%%%%%%%%%%%%%%%%

\colorlet{highlightbarcolor}{lightgray}
\colorlet{headerbarcolor}{darkgray}

\colorlet{headerfontcolor}{white}
\colorlet{accent}{awesome-red}
\colorlet{heading}{black}
\colorlet{emphasis}{black}
\colorlet{body}{black}


%%%%%%%%%%%%%%%%%%%%%%%%%%%%%%%%%%%%%%%%%%%%%%%%%%%%%%%%%%%%%%%%%%%%%%%%%%%%%%%%
% set document
%%%%%%%%%%%%%%%%%%%%%%%%%%%%%%%%%%%%%%%%%%%%%%%%%%%%%%%%%%%%%%%%%%%%%%%%%%%%%%%%


\begin{document}

\name{Mehdi BOUJA}
\tagline
{

Mon esprit créatif doublé de mes connaissances techniques et ma passion pour\\ les nouvelles technologies et les problèmes complexes font de moi un étudiant ingénieur aguerri qui a l'habitude de réagir et de prendre des décisions même sous une forte pression. \\ }
\photo[round]{}{\dimexpr \headerheight-\marginbottom}   % make photo exactly match the header with margintop/marginright/marginbottom as margin

\makeheader

\highlightbar{

    \section{Contact}
    
    \email{mehdibouja33@gmail.com}
    \phone{+33 7 58 09 57 54}
 	 \location{78 avenue du Colonel Fabien,}
    {93100  Montreuil}
    \vspace{0.5em}
    
    \linkedin{Mehdi Bouja}{https://www.linkedin.com/in/mehdi-bouja-52b078159/}
    
    
    \section{Compétences}
    
    \skillsection{Programmation}
    \textbf{Python}\\
   \textbf{ C/C++}\\
    \textbf{VBA}\\
    \textbf{VHDL}
   
    
    \vspace{0.5em}
    \skillsection{Systèmes d'Exploitation}
    \textbf{Linux}\\
    \textbf{MacOS}\\
    \textbf{Windows}
    
    \vspace{0.5em}
    \skillsection{Outils \& Logiciels}
   \textbf{ROS}\\
   \textbf{ Simulation}\\
    (e.g. Matlab/Simulink)\\
    \textbf{Conception}\\
    (e.g. SolidWorks,KiCad,Spice,Labview)\\
    \textbf{Office}
    
    \vspace{0.5em}
    \skillsection{Compétences Relationnelles}
    
\begin{itemize}
\item Résolution des problèmes.
\item     Travail d'équipe.
\item     Gestion du temps.
\item Flexibilité et Adaptabilité.
\end{itemize}
    
    \vspace{0.5em}
    \skillsection{Langues}
    \textbf{Arabe} : Langue maternelle\\
    \textbf{Français} : Bilingue\\
    \textbf{Anglais} : Avancé (TOEIC 920/990)\\
    \textbf{Espagnol} : Basique
      
      \vspace{0.5em}
     \skillsection{Intérêts}
   	
\begin{itemize}
\item Football.
\item Arts Martiaux (Kick-Boxing \& Taiekwondo).	
\item Voyage.
\end{itemize}
   
   \skillsection{A
   utres}
    
\begin{itemize}
\item Permis de conduire : Permis B
\item     Habilitation éléctrique niveau BE
\end{itemize}
    
    \bigskip
    
  
}
\mainbar{
    \section[\faGears]{Experience}
    \job{03/2022-Aujourd'hui}
        {THALES SIX GTS, \textit{PALAISEAU}}
        {Stage Ingénieur R\&D}
        {
\begin{itemize}
\item \textbf{Sujet}:Localisation des Drones Via Fusion de Capteurs Lidar.
\item \textbf{Objectif}:Développer une solution de localisation et de détection d'obstacles en fusionnant des données provenant de multiples capteurs lidar/caméras à champ de vision réduit.  \\
\end{itemize}
    
}
    \job{06/2021-08/2021}
        {GIAA, \textit{MADRID}}
        {Stage Ingénieur R\&D}
        {
\begin{itemize}
\item \textbf{Sujet}:Les Problèmes d'Atterrissage des Drones.
\item \textbf{Objectif}:Développer une solution au problème d'atterrissage des drones basée sur le traitement d'images et la vision par ordinateur . \\
\end{itemize}



    \job{06/2018-08/2018}
        {MAHASOFT, \textit{AGADIR}}
        {Assistant support informatique }
        {
\begin{itemize}
\item \textbf{Objectif}:Traiter toute une série de documents, saisir des informations dans la base de données de l'entreprise et établir les devis.

\end{itemize}
}
}
    \section[\faMortarBoard]{Formation}
    \job{09/2019 - 09/2022}
        {Telecom Physique Strasbourg,\\ \textit{STRASBOURG}}
        {Diplôme d'ingénieur généraliste}
        {
\begin{itemize}
\item \textbf{Option}:Ingénierie des Systèmes, Automatique et Vision.
\item \textbf{Cours pertinents}:Vision et commande, Robotique, Temps réel et systèmes embarqués, Réseaux industriels, Entrepreneuriat, Management d'équipe, Gestion financière.
\end{itemize}
}
	 \job{09/2020 - 09/2022}
        {Telecom Physique Strasbourg,\\ \textit{STRASBOURG}}
        {Master IRIV}
        {
\begin{itemize}
\item \textbf{Option}:Automatique et Robotique.
\item \textbf{Cours pertinents}:Robotique mobile, Traitement d'images, Formation des images, Vision 3D, Optimisation.
\end{itemize}
}
	 
	 

    \section{Projets}
    \job{2019- 2020}
        {Telecom Physique Strasbourg,\\ \textit{STRASBOURG}}
        {Projet Ingénieur}
        {
\begin{itemize}
\item \textbf{Objectif}:Reconstruction d'environnement par SLAM pour l'endoscopie flexible robotisée et commande des moteurs. 
}

\end{itemize}
    \job{2021- 2022}
        {Telecom Physique Strasbourg,\\ \textit{STRASBOURG}}
        {Projet BIP}
        {
\begin{itemize}
\item \textbf{Objectif}:Diagnostic des lésions cutanées à partir d'images dermoscopiques en utilisant des \\techniques de traitement d'images et d'IA.


\end{itemize}
    
    
    \job{2021- 2022}
        {Telecom Physique Strasbourg,\\ \textit{STRASBOURG}}
        {Hacking Industry Camp}
        {
\begin{itemize}
\item \textbf{Objectif}:Détection et suivi des chantiers de construction à l'aide d'images satellites.
         }

\end{itemize}
    
    \job{2020- 2021}
        {Telecom Physique Strasbourg,\\ \textit{STRASBOURG}}
        {Projet MiBot}
        {
\begin{itemize}
\item \textbf{Objectif} :Création d'un robot Segway afin de démontrer les capacités des nouveaux Circuits Intégrés de la société Allemande TRINAMIC lors des salons professionnels et dans des vidéos promotionnelles.
}

\end{itemize}

    
}
    \medskip
}
\makebody
\clearpage

\end{document}
